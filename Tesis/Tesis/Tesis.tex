\documentclass[11pt,spanish]{report}
\usepackage{biblatex}
\usepackage[spanish,mexico]{babel}
\usepackage{UNAMThesis}
\usepackage{amsmath}
\usepackage{amsfonts}
\usepackage{amssymb}

\usepackage[utf8]{inputenc}
\usepackage{csquotes}
\usepackage[T1]{fontenc}
\usepackage{lmodern}
\usepackage{textcomp}
\usepackage{array}
\usepackage{booktabs}

\logounam{Escudo-UNAM}
\logoinstitute{Escudo-Dependencia}
\pagenumbering{roman}
\flushbottom
\newtheorem{theorem}{Teorema}[section]
\newtheorem{acknowledgement}{Agradecimiento}[section]
\newtheorem{algorithm}{Algoritmo}[section]
\newtheorem{axiom}{Axioma}[section]
\newtheorem{case}{Caso}[section]
\newtheorem{claim}{Declaración}[section]
\newtheorem{conclusion}{Conclusión}[section]
\newtheorem{condition}{Condición}[section]
\newtheorem{conjecture}{Conjectura}[section]
\newtheorem{corollary}{Corolario}[section]
\newtheorem{criterion}{Criterio}[section]
\newtheorem{definition}{Definición}[section]
\newtheorem{example}{Ejemplo}[section]
\newtheorem{exercise}{Ejercicio}[section]
\newtheorem{lemma}{Lemma}[section]
\newtheorem{notation}{Notación}[section]
\newtheorem{problem}{Problema}[section]
\newtheorem{proposition}{Proposición}[section]
\newtheorem{remark}{Observación}[section]
\newtheorem{solution}{Solución}[section]
\newtheorem{summary}{Resumen}[section]
\newenvironment{proof}[1][Demostración]{\textbf{#1.} }{\ \rule{0.5em}{0.5em}}

\addbibresource{Bibliografia.bib}

\begin{document}

\title{Plantilla de tesis UNAM}
\author{Un autor con suerte}
\prevdegrees{Grados previos}
\prevdegrees{Grados previos}
\institute{Mi Instituto}
\department{Mi programa de estudios}
\degree{El grado que busco}
\supervisor{Mi tutor}
\city{Mi ciudad}
\degreemonth{Mes}
\degreeyear{Año}
\maketitle

\begin{dedication}
A pesar de la distancia y del\\
tiempo ido, sólo puedo dedicar:\\
A todos y ninguno...\\
\textsc{Julio A. Freyre-Gonz\'{a}lez}
\end{dedication}

\begin{acknowledgements}
Gracias a cada una de las personas que me apoyaron e hicieron que
este sueño se cristalizara en una hermosa realidad.
\end{acknowledgements}

\tableofcontents
\clearpage
\listoftables
\clearpage
\listoffigures
\clearpage

\begin{foreword}
Redacte aquí el prólogo contando la historia detrás de su
trabajo.
\end{foreword}

\begin{abstract}
Here goes the english abstract.
\end{abstract}

\begin{resumen}
Aquí se redacta el resumen en español.
\end{resumen}

\pagenumbering{arabic}

\part{Primera parte}

\chapter{Sobre el estílo}

Esto es un ejemplo para el paquete UNAMThesis, tiene un formato personalizable para tesis según las normas de la Universidad Nacional Autónoma de México.

Los datos de la portada son de ejemplo y deberían ser reemplazados.

\section{Uso del paquete}
Escudo-UNAM and Escudo-Instituto deben estar en la misma carpeta que el documento. El estilo busca estas imágenes para armar la carátula de la tesis. Se puede personalizar también los logos usando: \texttt{\textbackslash{}logounam\{lugar\_archivo\}} y \texttt{\textbackslash{}logoinstitute\{lugar\_archivo\}} para definir su ubicación.

El comando \texttt{\textbackslash{}university\{nombre de universidad\}} permite cambiar el nombre de la universidad, por defecto tiene el de la UNAM.

Además, los comandos \texttt{\textbackslash{}institute\{nombre de instituto\}}, \texttt{
\textbackslash{}rcenter\{nombre de centro de investigación\}},
\texttt{\textbackslash{}faculty\{nombre de facultad\}}, o
\texttt{\textbackslash{}school\{nombre de escuela\}} 
permiten establecer los nombres de donde se llevaron a cabo los estudios. La lógica de usar comandos diferentes es que se elija automáticamente el género correcto, pero los comandos son opcionales.

En caso de omitir el nombre del instituto, el género por defecto es femenito ('la') para coincidir con 'universidad'.
Se puede cambiar esto usando \texttt{\textbackslash{}@instituteartm} para cambiarlo a masculino ('el') o \texttt{\textbackslash{}@instituteartf} para cambiarlo a femenino
('la'). El comando \texttt{\textbackslash{}department\{nombre del programa del departamento\}}es opcional.


El estilo implementa también entornos para citas de aperturas
(\texttt{\textbackslash{}begin\{
quotenat\} ... \textbackslash{}end\{quotenat\}})
que se usan muchas veces al inicio de los capítulos.

El estilo bibliográfico de UNAMThesis es compatible con natbib.

\chapter{Características del estilo}

\begin{quotenat}
\textsl{I worry that, especially as the Millennium edges nearer,\\
pseudo-science and superstition will seem year by year more tempting,\\
the siren song of unreason more sonorous and attractive.}\\
--- \textsc{Carl Sagan, \textit{The Demon-Haunted World} (1995)}
\end{quotenat}

\section{Section}

Este estilo tiene muchas etapas de seccionamiento.

\subsection{Subsection}

Texto de subsection.

\subsubsection{Subsubsection}

Texto de subsubsection.

\paragraph{Paragraph}

Texto de paragraph.

\subparagraph{Subparagraph}

Texto de subparagraph.

\begin{table}
\centering
\caption[Tabla de ejemplo]{Tabla de ejemplo. Se muestran los elementos básicos que se usan al declarar una tabla, y se aprovecha para dar una breve descripción de las opciones de configuración.}
\label{tab:tabla_ejemplo}
\renewcommand{\arraystretch}{1.3}% Aumenta la separación entre renglones
\begin{tabular}{lm{0.5\linewidth}} \toprule
\multicolumn{1}{c}{\textbf{Valor}} & \multicolumn{1}{c}{\textbf{Descripción}} \\ \midrule
\textbf{l} & Columna alineada a la izquierda. \\
\textbf{c} & Columna centrada. \\
\textbf{r} & Columna alineada a la derecha. \\
\textbf{p\{'width'\}} & Columna de párrafo alineada verticalmente arriba. Útil para cuando hay mucho texto. \\
\textbf{m\{'width'\}} & Columna de párrafo alineada verticalmente en el centro. El estilo que usa esta tabla\\
\textbf{b\{'width'\}} & Columna de párrafo alineada verticalmente en la parte de abajo. \\
\textbf{|} & Línea vertical. \\
\textbf{||} & Doble línea vertical \\
\textbf{@\{xx\}} & \emph{xx} especifica el separador de la columna. \\ \bottomrule
\end{tabular}
\end{table}

\begin{figure}[htp]
\centering
\includegraphics[width=0.5\textwidth]{Escudo-UNAM.pdf}
\caption[Figura de ejemplo]{Figura de ejemplo. En la figura se observa el escudo de la UNAM que se usa en la portada, como una plantilla para el uso de imágenes. Se recomienda siempre usar imágenes vectoriales en formato PDF.}
\label{fig:figura_ejemplo}
\end{figure}


\section{Etiquetas}

Se puede aplicar diferentes formatos al texto como el de \emph{énfasis}

Y también se pueden aplicar diferentes estilos visuales como: \textbf{Bold}, \textit{Italics},
\textrm{Roman}, \textsf{Sans Serif}, \textsl{Slanted}, \textsc{Small Caps},
y \texttt{Typewriter}.

También se tienen estilos de tipografía para entornos matemáticos: $\mathbb{BLACKBOARD}$
$\mathbb{BOLD}$, $\mathcal{CALLIGRAPHIC}$, y $\mathfrak{fraktur}$. Hay que notar que el \emph{blackboard} y el \emph{calligraphic} solo se aplican correctamente a letras mayúsculas.

Se pueden aplicar las etiquetas de tamaño: {\tiny tiny}, {\scriptsize scriptsize},
{\footnotesize footnotesize}, {\small small}, {\normalsize normalsize},
{\large large}, {\Large Large}, {\LARGE LARGE}, {\huge huge} y {\Huge Huge}.

Se muestra a continuación un grupo de párrafos macadas como citas (quote):

\begin{quote}
The buck stops here. \emph{Harry Truman}

Ask not what your country can do for you; ask what you can do for your
country. \emph{John F Kennedy}

I am not a crook. \emph{Richard Nixon}

It's no exaggeration to say the undecideds could go one way or another.
\emph{George Bush}

I did not have sexual relations with that woman, Miss Lewinsky. \emph{Bill Clinton}
\end{quote}

La etiqueta de \emph{quotation} se usa para citas que ocupan más de un párrafo, a continuación está el inicio de emph{Alice's Adventures in Wonderland } por
Lewis Carroll:

\begin{quotation}
Alice was beginning to get very tired of sitting by her sister on the bank,
and of having nothing to do: once or twice she had peeped into the book her
sister was reading, but it had no pictures or conversations in it, 'and what
is the use of a book,' thought Alice 'without pictures or conversation?'

So she was considering in her own mind (as well as she could, for the hot day
made her feel very sleepy and stupid), whether the pleasure of making a
daisy-chain would be worth the trouble of getting up and picking the daisies,
when suddenly a White Rabbit with pink eyes ran close by her.

There was nothing so very remarkable in that; nor did Alice think it so very
much out of the way to hear the Rabbit say to itself, 'Oh dear! Oh dear! I
shall be late!' (when she thought it over afterwards, it occurred to her that
she ought to have wondered at this, but at the time it all seemed quite
natural); but when the Rabbit actually took a watch out of its
waistcoat-pocket, and looked at it, and then hurried on, Alice started to her
feet, for it flashed across her mind that she had never before seen a rabbit
with either a waistcoat-pocket, or a watch to take out of it, and burning with
curiosity, she ran across the field after it, and fortunately was just in time
to see it pop down a large rabbit-hole under the hedge.

In another moment down went Alice after it, never once considering how in the
world she was to get out again.
\end{quotation}

Se puede usar el entorno \emph{verbatim} cuando se quiere que \LaTeX {} conserve los espacios, como en un fragmento de código de programa:

\begin{verbatim}
#include <iostream>        // < > is used for standard libraries.
void main(void)            // ''main'' method always called first.
{
  cout << ''Hello World.'';  // Send to output stream.
}
\end{verbatim}

\section{Matemáticas y texto}

Sea $H$ un espacio de Hilbert, $C$ un subespacio cerrado convexo de $H$, $T$
mapa no-expansivo propio de $C$. Suponemos que conforme $n\rightarrow\infty$,
$a_{n,k}\rightarrow0$ para cada $k$, y $\gamma_{n}=\sum_{k=0}^{\infty}\left(
a_{n,k+1}-a_{n,k}\right)  ^{+}\rightarrow0.$ Entonces para cada $x$ en $C$,
$A_{n}x=\sum_{k=0}^{\infty}a_{n,k}T^{k}x$ converge suavemente a un punto fijo de $T$ .

La ecuación enumerada
\begin{equation}
u_{tt}-\Delta u+u^{5}+u\left|  u\right|  ^{p-2}=0\text{ in }\mathbf{R}
^{3}\times\left[  0,\infty\right[  .\label{eqn1}
\end{equation}
Se numera automáticamente como \ref{eqn1}.

\section{Lista de entornos}

Se pueden crear listas de viñetas y números y descripciones.

\begin{enumerate}
\item Elemento 1

\item Elemento 2

\begin{enumerate}
\item Lista enumerada anidada.

\item Otro elemento de la lista enumerada.

\begin{enumerate}
\item Tercer nivel de la lista.

\begin{enumerate}
\item Cuarto y último nivel de la lista.
\end{enumerate}
\end{enumerate}
\end{enumerate}
\end{enumerate}

\begin{itemize}
\item Lista de viñetas

\item Segundo elemento con viñetas

\begin{itemize}
\item Segundo nivel de la viñeta

\begin{itemize}
\item tercer nivel de la lista de viñetas.

\begin{itemize}
\item Cuarto y último nivel de la lista anidada.
\end{itemize}
\end{itemize}
\end{itemize}
\end{itemize}

\begin{description}
\item[Lista descriptiva] Cada lista descriptiva tiene un elemento con un término seguido de la descripción de dicho término.

\item[Bunyip] Bestia mítica de las leyendas Australianas.
\end{description}

\section{Entornos como teoremas}

El estilo incluye muchos entornos que se comportan como teoremas:

\begin{acknowledgement}
Esto es un agradecimiento
\end{acknowledgement}

\begin{algorithm}
Esto es un algoritmo
\end{algorithm}

\begin{axiom}
Esto es un axioma
\end{axiom}

\begin{case}
Esto es un caso
\end{case}

\begin{claim}
Esto es una declaración
\end{claim}

\begin{conclusion}
Esto es una conclusión
\end{conclusion}

\begin{condition}
Esto es una condición
\end{condition}

\begin{conjecture}
Esto es una conjetura
\end{conjecture}

\begin{corollary}
Esto es un corolario
\end{corollary}

\begin{criterion}
Esto es un criterio
\end{criterion}

\begin{definition}
Esto es una definición
\end{definition}

\begin{example}
Esto es un ejemplo
\end{example}

\begin{exercise}
Esto es un ejercicio
\end{exercise}

\begin{lemma}
Esto es un lemma
\end{lemma}

\begin{proof}
Esto es la demostración.
\end{proof}

\begin{notation}
Esto es una notación
\end{notation}

\begin{problem}
Esto es un problema
\end{problem}

\begin{proposition}
Esto es una proposición
\end{proposition}

\begin{remark}
Esto es una observación
\end{remark}

\begin{solution}
Esto es una solución
\end{solution}

\begin{summary}
Esto es un resumen
\end{summary}

\begin{theorem}
Esto es un teorema
\end{theorem}

\begin{proof}
[Demostración]Esta es la demostración.
\end{proof}

\appendix 

\chapter{El primer apéndice}

La instrucción \emph{appendix} solo se usa una vez. Apéndices posteriores se pueden crear usando las instrucciones de \emph{chapter}, \emph{section} y demás.

\printbibliography
\end{document}