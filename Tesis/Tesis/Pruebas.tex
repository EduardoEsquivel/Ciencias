\documentclass[10pt,spanish]{book}
\usepackage{amsmath}
\usepackage[spanish,mexico]{babel}
\usepackage{bookman}
\usepackage{etoolbox}
\usepackage[T1]{fontenc}
\usepackage[lmargin=3.0cm, rmargin=1.0cm,tmargin=2.50cm,bmargin=2.50cm]{geometry}
\usepackage[utf8]{inputenc}
\usepackage{lipsum}
\usepackage{mathtools}
\usepackage{titlesec}
\usepackage{xcolor}

%\ChNameVar{\large\fontfamily{pbk}\selectfont
%\color{black}}
%\ChRuleWidth{2pt}
%\ChNumVar{\large\fontfamily{pbk}\selectfont
%\color{black}}
%\ChTitleVar{\Large\bfseries\fontfamily{phv}\selectfont\scshape\color{black}}

\titleformat{\chapter}[display]
{\large\fontfamily{pvh}\filleft}
{\titlerule[0pt]%
\vspace{1ex}% 
\chaptertitlename\ \thechapter}
{20pt}
{\Huge}[\vspace{1ex}{\titlerule[0pt]}]

\titleformat{name=\chapter,numberless}[display]
{\fontfamily{pbk}\Huge\filleft\bfseries}
{}
{0pt}
{\titlerule[2pt]
\vspace{1ex}%
\Huge}[\vspace{1ex}{\titlerule[2pt]}]

\titlespacing*{\chapter} {0pt}{20pt}{20pt}   %% adjust these numbers
\titlespacing*{name=\chapter,numberless} {0pt}{20pt}{20pt}   %% adjust these n

\begin{document}

\chapter{Resonadores ópticos}

%\ChNameUpperCase{CAPÍTULO}
Esta es una ecuación en un ambiente.
\begin{equation}
x^{2} = -1
\end{equation}

Esta es una ecuación en línea \,\, \( x^{2} = -1\) \newline

Esta es una ecuación desplegada \[ x^{2} = -1\]

Esta es una ecuación desplegada y encerrada en un rectángulo

\[ \boxed{ x^{2} = -1 } \]

Esta es una ecuación desplegada, enumerada y encerrada en un rectángulo
\begin{equation}
\boxed{ x^{2} = -1 }
\end{equation}

De esta forma hacemos un salto de \\ línea 
\newline
\lipsum

\end{document}